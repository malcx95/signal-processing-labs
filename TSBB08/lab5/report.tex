\documentclass[twocolumn]{article}
\usepackage[utf8]{inputenc}
\usepackage{newtxtext}
\usepackage{newtxmath}
\usepackage{spverbatim}
\usepackage{graphicx}
\usepackage{icomma}
\usepackage{listings}
\usepackage{xcolor}
\usepackage{color}
\usepackage[titletoc,toc,title]{appendix}
\definecolor{dkgreen}{rgb}{0,0.6,0}
\definecolor{gray}{rgb}{0.5,0.5,0.5}
\definecolor{mauve}{rgb}{0.58,0,0.82}

\lstset{%
    aboveskip=3mm, belowskip=3mm,
    showstringspaces=false,
    columns=flexible,
    basicstyle={\small\ttfamily},
    numbers=none,
    numberstyle=\tiny\color{red},
    keywordstyle=\color{blue},
    commentstyle=\color{dkgreen},
    stringstyle=\color{mauve},
    breaklines=true,
    breakatwhitespace=true,
    tabsize=3
}

\raggedbottom
\sloppy

\title{Report for Lab 5 -- TSBB08 \emph{Digital Image Processing}}

\author{Malcolm Vigren \\ malvi108, 19950127--0970 }

\date{\today}

\begin{document}

\maketitle

\section{Loading image and separating channels}

The first step was to load the original image (Figure \ref{fig:origim}).

\begin{figure}[h!]
    \centering
    \includegraphics[width=0.7\linewidth]{images/origim.jpg}
    \caption{Original image}
    \label{fig:origim}
\end{figure}

The red, green and blue channels are isolated, see
Figures~\ref{fig:origimr},~\ref{fig:origimg} and~\ref{fig:origimb}.

\begin{figure}[h!]
    \centering
    \includegraphics[width=0.7\linewidth]{images/origimr.jpg}
    \caption{Red channel of original image}
    \label{fig:origimr}
\end{figure}

\begin{figure}[h!]
    \centering
    \includegraphics[width=0.7\linewidth]{images/origimg.jpg}
    \caption{Green channel of original image}
    \label{fig:origimg}
\end{figure}

\begin{figure}[h!]
    \centering
    \includegraphics[width=0.7\linewidth]{images/origimb.jpg}
    \caption{Blue channel of original image}
    \label{fig:origimb}
\end{figure}

We can from the separated channels see that the nuclei are entirely contained
in the blue channel and the padlocks entirely in the red channel. This allows
us to process two separate images when determining the size of the cytoplasms
and the number of padlocks.

\section{Creating binary image of nuclei}

Since the nuclei were contained in the blue channel of the original image,
we calculate the histogram of the blue channel, see Figure~\ref{fig:hist}.

\begin{figure}[h!]
    \centering
    \includegraphics[width=0.7\linewidth]{images/hist.jpg}
    \caption{Histogram of the blue channel}
    \label{fig:hist}
\end{figure}

From this histogram we can see that 50 is a suitable threshold value, so the image
is thresholded with this value, see Figure~\ref{fig:thld}.

\begin{figure}[h!]
    \centering
    \includegraphics[width=0.7\linewidth]{images/thld.jpg}
    \caption{The thresholded blue channel}
    \label{fig:thld}
\end{figure}

The threshold turned out quite well, but there is still some unwanted small objects
in the image. This is solved by eroding the thresholded image twice, and then
dilating it twice. The result of this is shown in Figure~\ref{fig:erdi}.

\begin{figure}[h!]
    \centering
    \includegraphics[width=0.7\linewidth]{images/erdi.jpg}
    \caption{Result after eroding and dilating}
    \label{fig:erdi}
\end{figure}

\section{Segmentation of nuclei}

In Figure~\ref{fig:erdi}, some of the nuclei are connected to eachother. These
need to be segmented.

First, a distance map is calculated from the inverted binary image. The sign of
the distance map is changed, which creates regional min-points inside the
nuclei. This is shown in Figure~\ref{fig:distmap}.

\begin{figure}[h!]
    \centering
    \includegraphics[width=0.7\linewidth]{images/distmap.jpg}
    \caption{Distance map, where blue is more negative}
    \label{fig:distmap}
\end{figure}

Unfortunately, one of the pairs of nuclei are too close to eachother, creating
a regional min-point at their intersection, which would later lead to
oversegmentation. This is fixed by suppressing local minima that have a depth
less than 30, shown in Figure~\ref{fig:suppressed}.

\begin{figure}[h!]
    \centering
    \includegraphics[width=0.7\linewidth]{images/suppressed.jpg}
    \caption{Distance map with suppressed local minima}
    \label{fig:suppressed}
\end{figure}

Now the min-points can be marked (Figure~\ref{fig:minpoints}) and labeled
(Figure~\ref{fig:labeled}).

\begin{figure}[h!]
    \centering
    \includegraphics[width=0.7\linewidth]{images/minpoints.jpg}
    \caption{Marked min-points}
    \label{fig:minpoints}
\end{figure}

\begin{figure}[h!]
    \centering
    \includegraphics[width=0.7\linewidth]{images/labeled.jpg}
    \caption{The min-points labeled}
    \label{fig:labeled}
\end{figure}

The watershed of the image can now be calculated (Figure~\ref{fig:wshednuc}) 
and converted to binary image (Figure~\ref{fig:binwshednuc}).

\begin{figure}[h!]
    \centering
    \includegraphics[width=0.7\linewidth]{images/wshednuc.jpg}
    \caption{The calculated watershed}
    \label{fig:wshednuc}
\end{figure}

\begin{figure}[h!]
    \centering
    \includegraphics[width=0.7\linewidth]{images/binwshednuc.jpg}
    \caption{Final segmented nuclei}
    \label{fig:binwshednuc}
\end{figure}

\section{Segmentation of cytoplasms}

Now the cytoplasms need to be segmented. This is done by first computing the distance
map outside the nuclei and limit the maximum distance to 100 pixels, shown in
Figure~\ref{fig:distcyt}.

\begin{figure}[h!]
    \centering
    \includegraphics[width=0.7\linewidth]{images/distcyt.jpg}
    \caption{Distance map of the cytoplasms}
    \label{fig:distcyt}
\end{figure}

The watershed is now calculated. To make sure that the cytoplasms are bounded,
a pixel outside the cells is set to 1. This creates a "waterhole", which will
enable "the sea" to be filled with water in the watershed segmentation. The
computed watershed is shown in Figure~\ref{fig:wshedcyt}.

\begin{figure}[h!]
    \centering
    \includegraphics[width=0.7\linewidth]{images/wshedcyt.jpg}
    \caption{The watershed segmented cytoplasms}
    \label{fig:wshedcyt}
\end{figure}

\section{Creating the border}

The border was created from the watershed of the cytoplasms by first observing
that the border in that image had value 0. Then all pixels with the value 0 in
that image was changed to 1, and all others to 0. This border was then dilated
one iteration to make it slightly thicker. Figure~\ref{fig:border} shows the
result of this.

\begin{figure}[h!]
    \centering
    \includegraphics[width=0.7\linewidth]{images/border.jpg}
    \caption{The border of the cytoplasms}
    \label{fig:border}
\end{figure}

\section{Isolating a particular cell}

To isolate one of the cells, first the section we want to isolate is identified
in the watershed image. Since the region has the value 6, we set positions
where the watershed image isn't 6 to 0 in the original image. We also multiply
the green channel with 0, since we are not interested in anything in there. The
result of this is shown in Figure~\ref{fig:isolated}.

\begin{figure}[h!]
    \centering
    \includegraphics[width=0.7\linewidth]{images/isolated.jpg}
    \caption{The isolated cell with corresponding padlocks}
    \label{fig:isolated}
\end{figure}

\section{Circling padlock cells}

In order to identify each padlock cell in the image, we first filter the red
channel of the isolated image using a negative Laplace filter. This will 
enhance the padlocks, since the filter is essentially a matching filter for 
such objects. The result of the filtering is shown in
Figure~\ref{fig:filtered}.

\begin{figure}[h!]
    \centering
    \includegraphics[width=\linewidth]{images/filtered.jpg}
    \caption{The laplace filtered padlocks}
    \label{fig:filtered}
\end{figure}

From this, we can easily identify regional maxima, shown in
Figure~\ref{fig:maxima}.

\begin{figure}[h!]
    \centering
    \includegraphics[width=\linewidth]{images/maxima.jpg}
    \caption{Max-points from the filtered padlocks}
    \label{fig:maxima}
\end{figure}

To circle these, we create grid of zeros and ones, where the ones resemble a
circle. We then overlay these circles on a blank image in the same
positions as the padlocks. Figure~\ref{fig:circles} shows these circles.

\begin{figure}[h!]
    \centering
    \includegraphics[width=\linewidth]{images/circles.jpg}
    \caption{Circles in the padlock positions}
    \label{fig:circles}
\end{figure}

\section{Creating the final image}

Now we combine the cytoplasm border and the padlock circles with the original
image. To make the border yellow, we put it in the red and green channels of
the original image. To make the circles cyan, we put the circles image in the
green and blue channels. Since both the circles and border need to be in the
green channel, we combine these before adding them to the image. The final
result is shown in Figure~\ref{fig:final}.

\onecolumn
\begin{figure}[h!]
    \centering
    \includegraphics[width=\linewidth]{images/final.jpg}
    \caption{The final image}
    \label{fig:final}
\end{figure}

\newpage
\onecolumn
\begin{appendices}

\section{The MATLAB code}

\subsection{The main code}

\lstinputlisting[language=matlab]{Main.m}

\subsection{The code for countpadlocks}

\lstinputlisting[language=matlab]{countpadlocks.m}

\end{appendices}

\end{document}
