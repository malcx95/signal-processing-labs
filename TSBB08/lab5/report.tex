\documentclass[twocolumn]{article}
\usepackage[utf8]{inputenc}
\usepackage{newtxtext}
\usepackage{newtxmath}
\usepackage{spverbatim}
\usepackage{graphicx}
\usepackage{icomma}
\usepackage{listings}
\usepackage{xcolor}
\usepackage{color}
\usepackage[titletoc,toc,title]{appendix}
\definecolor{dkgreen}{rgb}{0,0.6,0}
\definecolor{gray}{rgb}{0.5,0.5,0.5}
\definecolor{mauve}{rgb}{0.58,0,0.82}

\lstset{%
    aboveskip=3mm, belowskip=3mm,
    showstringspaces=false,
    columns=flexible,
    basicstyle={\small\ttfamily},
    numbers=none,
    numberstyle=\tiny\color{red},
    keywordstyle=\color{blue},
    commentstyle=\color{dkgreen},
    stringstyle=\color{mauve},
    breaklines=true,
    breakatwhitespace=true,
    tabsize=3
}

\raggedbottom
\sloppy

\title{Report for Lab 5 -- TSBB08 \emph{Digital Image Processing}}

\author{Malcolm Vigren \\ malvi108, 19950127--0970 }

\date{\today}

\begin{document}

\maketitle

\section{Loading image and separating channels}

The first step was to load the original image (Figure \ref{fig:origim}).

\begin{figure}[h!]
    \centering
    \includegraphics[width=0.7\linewidth]{images/origim.jpg}
    \caption{Original image}
    \label{fig:origim}
\end{figure}

The red, green and blue channels are isolated, see
Figures~\ref{fig:origimr},~\ref{fig:origimg} and~\ref{fig:origimb}.

\begin{figure}[h!]
    \centering
    \includegraphics[width=0.7\linewidth]{images/origimr.jpg}
    \caption{Red channel of original image}
    \label{fig:origimr}
\end{figure}

\begin{figure}[h!]
    \centering
    \includegraphics[width=0.7\linewidth]{images/origimg.jpg}
    \caption{Green channel of original image}
    \label{fig:origimg}
\end{figure}

\begin{figure}[h!]
    \centering
    \includegraphics[width=0.7\linewidth]{images/origimb.jpg}
    \caption{Blue channel of original image}
    \label{fig:origimb}
\end{figure}

We can from the separated channels see that the nuclei are entirely contained
in the blue channel and the padlocks entirely in the red channel. This allows
us to process two separate images when determining the size of the cytoplasms
and the number of padlocks.

\section{Creating binary image of nuclei}

Since the nuclei were contained in the blue channel of the original image,
calculate the histogram of the blue channel, see Figure~\ref{fig:hist}.

\begin{figure}[h!]
    \centering
    \includegraphics[width=0.7\linewidth]{images/hist.jpg}
    \caption{Histogram of the blue channel}
    \label{fig:hist}
\end{figure}

From this histogram we can see that 50 is a suitable threshold value, so the image
is thresholded with this value, see Figure~\ref{fig:thld}.

\begin{figure}[h!]
    \centering
    \includegraphics[width=0.7\linewidth]{images/thld.jpg}
    \caption{The thresholded blue channel}
    \label{fig:thld}
\end{figure}

The threshold turned out quite well, but there is still some unwanted small objects
in the image. This is solved by eroding the thresholded image twice, and then
dilating it twice. The result of this is shown in Figure~\ref{fig:erdi}.

\begin{figure}[h!]
    \centering
    \includegraphics[width=0.7\linewidth]{images/erdi.jpg}
    \caption{Result after eroding and dilating}
    \label{fig:erdi}
\end{figure}

\section{Segmentation of nuclei}

In Figure~\ref{fig:erdi}, some of the nuclei are connected to eachother. These
need to be segmented.

First, a distance map is calculated from the inverted binary image. The sign of
the distance map is changed, which creates regional min-points inside the
nuclei. This is shown in Figure~\ref{fig:distmap}.

\begin{figure}[h!]
    \centering
    \includegraphics[width=0.7\linewidth]{images/distmap.jpg}
    \caption{Distance map, where blue is more negative}
    \label{fig:distmap}
\end{figure}

Unfortunately, one of the pairs of nuclei are too close to eachother, creating
a regional min-point at their intersection, which would later lead to
oversegmentation.

\newpage
\onecolumn
\begin{appendices}

\section{The MATLAB code}

    \lstinputlisting[language=matlab]{Main.m}

\end{appendices}

\end{document}
