\documentclass[a4paper]{article}
\usepackage[utf8]{inputenc} %Make sure all UTF8 characters work in the document
\usepackage{listings} %Add code sections
\usepackage{color}
\usepackage{graphicx}
\usepackage{titling}
\usepackage{textcomp}
\usepackage[hyphens]{url}
\usepackage[bottom]{footmisc}
\definecolor{listinggray}{gray}{0.9}
\definecolor{lbcolor}{rgb}{0.9,0.9,0.9}

%Set page size
\usepackage{geometry}
\geometry{margin=2cm}
\usepackage{parskip} 
%\pretitle{%	en bild för framsidan
	%\begin{center}
	%\LARGE
%	\includegraphics[width=6cm]{python.png}\\[\bigskipamount]
%}
\title{Peer-review of report 76}
\renewcommand*\contentsname{Innehållsförteckning}
\begin{document}
\maketitle

\section{Overall comments}

The report is easy to read and understand, and the MATLAB code is nicely structured and
readable. However, the report is missing some important results, and some parts
need clarification. These points points are described below, section by
section.

\section{Introduction}

The introduction is rather brief. We personally would expect the introduction
to contain some information of what the different parts of the study were.

\section{Part 1 -- Whistle}

\begin{itemize}
    \item There is a formatting mistake on the word "kHz". You wrote $kH_z$, when
it should be "kHz". 
    \item Figure 1 has an incorrect label and grading of the horizontal axis.
        It could be in seconds or samples, not Hz since we are looking in the
        time-domain.
\end{itemize}

\subsection{Tasks and implementation}

\begin{itemize}
    \item There is a formatting mistake on "$H_dist$", it should be $H_{dist}$
    \item Would be nice with some formulas, reference to the book or reference to
    the code for how the parameters of the AR-model are estimated.
    \item You wrote about "a knee in the graph". This is a bit unclear, it
        would be better to show this graph or explain it in a different way.
\end{itemize}

\subsection{Results}
\begin{itemize}
    \item You wrote that the dominating frequency is [X, Y]. It doesn't
        say anywhere what X and Y are.
    \item As with Figure 1, Figure 3 has an incorrect label and grading of the horizontal axis.
    \item You should write about what the dominating frequency was.
    \item It would be nice with more discussion about what these results mean.
\end{itemize}

\section{Part 2 -- Vowel}

\begin{itemize}
    \item There is a formatting mistake on the word "kHz".
\end{itemize}

\subsection{Tasks and implementation}

\begin{itemize}
    \item It would be nice with a reference to Figure 5 when explaining how the
    order of the model is determined.
    \item You write about how you determine the whiteness of the residual, but
        not why you would want to do that.
    \item You briefly describe at the end a second validation method, but it
        is not clear how and why the method works. Formulas, references to the
        course book or to where this is done in the code would help clarify
        what you mean.
\end{itemize}

\subsection{Results}

\begin{itemize}
    \item In Figure 6, you should change the labels of the
        horizontal axis to "order", as in Figure 5.
    \item It is not clear what Figure 7 and 8 are showing. 
        More background information about this would be helpful.
    \item You should present which model orders you selected, and
        the results of the simulation of the vowels.
    \item The meaning of the results should be discussed.
    \item There is no reference to Equation 4.
\end{itemize}

\section{Speech}

\begin{itemize}
    \item You wrote that you divide the signal into 160 segments.
        You were supposed to divide the signal into segments of 160
        samples each.
    \item You write about "theta values", without an explanation
        of what these are. An equation or rewording would clarify this.
    \item You write about finding the maximum of the covariance functions
        of the residual errors,
        but you should clarify that you don't consider values close
        to the origin.
    \item You should write about the task of setting the amplitude
        of all pulses in the pulse train to 1, since you report
        how this went in the results.
\end{itemize}

\subsection{Results}

\begin{itemize}
    \item You wrote that you set the amplitude of all segments to 1,
        what is actually done is that you set the amplitude of the
        input pulse trains to 1.
    \item You wrote that the AR(8)-model needs 9 values, but it
        only needs 8, since $a_0$ is always 1. You should therefore
        check what you wrote about the compression factor.
    \item You should include a discussion of that the result means.
\end{itemize}

\section{Matlab code}

\begin{itemize}
    \item There are lines in the MATLAB-code that are too long and
        do not fit the page.
    \item You should add references to your code in the text.
\end{itemize}

\section{Other comments}
\begin{itemize}
    \item Equation punctuation is not consistent in the report.
        Sometimes you have a period at the end of the sentence
        before an equation and sometimes you do not. We don't
        think there is a rule about this, but you shoud at least
        try to be consistent. In the course book they seem to prefer
        to have the equations as a part of the text and then have the
        period after the equations.
    \item Some parts of the the text were hard to read and understand at
        first read, and it contains a few minor formatting, spelling and
        grammatical mistakes. A proof read of the text would help.
\end{itemize}

\end{document}
